
\documentclass[12pt,a4paper,roman]{moderncv}        % possible options include font size ('10pt', '11pt' and '12pt'), paper size ('a4paper', 'letterpaper', 'a5paper', 'legalpaper', 'executivepaper' and 'landscape') and font family ('sans' and 'roman')
\usepackage{fancyhdr}
\usepackage{lastpage}
\pagestyle{fancy}
\fancyfoot{}
\cfoot{ \thepage\ / \pageref{LastPage}}
%\chead{ \thepage\ / \pageref{LastPage}}
\pagenumbering{roman}
\moderncvstyle{classic}                             % style options are 'casual' (default), 'classic', 'banking', 'oldstyle' and 'fancy'
\moderncvcolor{blue}                               % color options 'black', 'blue' (default), 'burgundy', 'green', 'grey', 'orange', 'purple' and 'red'
%\renewcommand{\familydefault}{\sfdefault}         % to set the default font; use '\sfdefault' for the default sans serif font, '\rmdefault' for the default roman one, or any tex font name

\usepackage[utf8]{inputenc}
\usepackage[scale=.75]{geometry}
%\setlength{\hintscolumnwidth}{3cm}
%\setlength{\makecvtitlenamewidth}{10cm}

\name{Zhongcheng}{Dai}
%\title{universität duisburg-essen}
\address{Kommandantenstr.57}{Duisburg 47057}
\phone[mobile]{015166267483}
\email{zhongcheng.dai@outlook.com}
%\social[twitter]{www.seccsc.com}
\homepage{www.seccsc.com}
\photo[80pt][0.4pt]{zc.jpg}      
\quote{"The only way to do great work is to love what you do." - Steve Jobs}
\makeatletter\renewcommand*{\bibliographyitemlabel}{\@biblabel{\arabic{enumiv}}}\makeatother
\begin{document}
\makecvtitle
\section{Ausbildung}
\cventry{2017--2019}{Master of Engineering in Automation and Control Engineering}{Universität Duisburg-Essen}{}{Elektrotechnik und informationstechnik}{Automatisierungstechnik und komplexe Systeme} 
\cventry{2016--2017}{Deutsch lernen}{Communikation Akoun \& Scholten}{}{}{}
\cventry{2011--2016}{Bachelor's Degree in Electrical Engineering and Automation}{Shandong University of Science and Technology}{}{}{Fakultät für Informations- und Elektrotechnik}
\vspace{0.35cm}
\section{Hauptkurs und Abschlussarbeit}
\cventry{2017-2019}{Eine Fehlererkennung mit SVM-Technik basierend auf $T^2$-Statistik und Riemannschen Mannigfaltigkeiten
}{}{}{}{
\begin{itemize}
    \item Robust Control Theory
    \item Prozessautomatisierung
    \item Nonlinear Control Systems
    \item Theorie statistischer Signale
    \item Mehrgrößen Regelungstechnik
    \item State and Parameter Estimation
    \item Modelling and Simulation of Dynamic Systems
    \item Fehlerdiagnose und Fehlertoleranz in technischen System
\end{itemize}
}

%\subsection{Work Experience}

%\cventry{2018--2019}{Porter}{Kunes Country Automotive Group}{Woodstock, IL}{}{}

%\cventry{2017--2018}{Chef}{Culver's}{Rockford, IL}{}{}

%\cventry{2017}{Christmas Tree Farm Sales}{Nature's Best Tree Farm}{Poplar Grove, IL}{}{}

%\cventry{2014--2016}{Pizza Cook}{Slice of Italy}{Rockford, IL}{}{}
\vspace{0.35cm}
\section{PC-Kenntnisse}
\cvitem{\textbf{OS}}{Linux \hspace{1.3cm}Microsoft Windows}
\cvitem{\textbf{Werkzeuge}}{Matlab/Simulink \hspace{1.3cm} Vim \hspace{1.3cm}Git}
\cvitem{\textbf{P.sprache}}{C\hspace{1.3cm}Python\hspace{1.3cm}VHDL\hspace{1.3cm}\LaTeX\hspace{1.3cm}Markdown}

\section{Praktikum und Projekt}
%%%%%%%\end{itemize}}
\cvitem{11.2018-01.2019}{Adaptive Steuerung des beheizten R\"uhrbeh\"alter
\begin{itemize}
\item Ziel: Steuerung f\"ur den F\"ullstand und den Temperatur. Verleichung des
Resultats des PID-Reglers und des Adaptive Reglers
\item Aufgabe: Entwurf des Reglers und Beweisung den Algorithmen auf dem
Tank.
\end{itemize}
}
\cvitem{07.2018-08.2018}{Pulsweitenmodulierte Regelung am Drei-Tank-System
\begin{itemize}
\item Ziel: Regelung des  F\"ullstand der Drei-Tank-System mit Hilfe der HiLSimulation
und Pulsweitenmodulation (SPS)
\item Aufgabe: Aufbau des mathematischen System. Entkoppelung und
Linearisierung. Entwurf des PI-Reglers.
\end{itemize}
}
\cvitem{07.2018-06.2018}{k\"unstliche neuronale Netz
\begin{itemize}
\item Ziel: Im ersten Teil der Praktikumsanleitung werden die mathematischen
Grundlagen zum Verst\"andnis k\"unstlicher Neuronen und neuronaler Netze
bereitgestellt. Im zweiten Teil wird das lernverfahren „Backpropagation“
hergeleitet und die damit verbundenen Probleme erläutert. Im dritten Teil wird das Perzeptron beschrieben.
\item Aufgabe: Handschriftliche Ziffern erkennen
\end{itemize}
}
\cvitem{05.2017-06.2017}{Zustandsbeobachtung am Invertierten Pendel
\begin{itemize}
\item Ziel: Entwurf des Luenberger Beobachters. Einsetzung den Algorithmen zur Zustandsbeobachtung am Labormodell „invertiertes Pendel“.
\item Aufgabe: Aufbau und Linearisierung des mathematischen Systems in
Zustandsraumdarstellung. Entwurf des Reglers durch die Polvorgabe und
das Vorfilter.
\end{itemize}
}
\cvitem{04.2017-05.2017}{Zustands- und Fuzzy-Regelung an Ball und Wippe
\begin{itemize}
\item Ziel: Entwurf der Zustandsregelung, des Fuzzy-Reglers und des
Reduzierbeobachters. 
\item Aufgabe: Aufbau des mathematische Modells  Entwurf der Abtastregelung mit R\"uckf\"uhrung
des Zustandsvektor und Zustandsbeobachter im Regelkreis.
\end{itemize}
}
\cvitem{04.2014-05.2014}{Biped Robot Contest
\begin{itemize}
\item Ziel: Fahren geradeaus und biegen Sie an der angegebenen Stelle nach vorne ab. Der schnellste gewinnt.
\item Aufgabe: Programmieren in C-Sprache, Entwerfen von Steuermodulen, Entwerfen von Antriebsmotormodulen.
\end{itemize}
}
%\section{Relevant Classes}
%%%\cvitem{Other}{Entrepreneurship}

%\section{High School Activities}
%%\cvlistitem{Varsity Football Kicker/Punter}

%\section{College Activities}
%\cvlistitem{Varsity Football Kicker}

%\section{Community Activities}
%%%\cvlistitem{Tutoring: Various Students in Sciene and Math}

%\vspace{2mm}

%%%%%%%%%%%%%%%%%%%%%%%\end{cvcolumns}

\end{document}









